\documentclass[a4paper, oneside, 12pt]{extarticle}
\usepackage[utf8]{inputenc}
\usepackage[ngerman]{babel}
\usepackage[top=3cm, bottom=2cm, outer=3cm, inner=3cm, heightrounded]{geometry}
\usepackage{graphicx}
\usepackage{morefloats}
\usepackage{wrapfig}
\usepackage{hyperref}
\usepackage{cite}
\usepackage{siunitx}
\usepackage[default]{sourcesanspro}
\usepackage[T1]{fontenc}

\usepackage{url}

\usepackage{marginnote}
\renewcommand*{\marginfont}{\noindent\rule{0pt}{0.7\baselineskip}\footnotesize}
\usepackage[font=footnotesize]{caption}

\usepackage{color}
\usepackage{xcolor}

\usepackage{multicol}

\usepackage{mathtools}
\usepackage{amssymb}

\usepackage{wrapfig}
\usepackage{ragged2e}

%%%SECTIONING
\usepackage[noindentafter]{titlesec}

\newcommand{\aufgabe}[1]{\section{#1}}

\newcommand{\zusatzaufgabe}[1]{\subsection{#1}}

\renewcommand{\theenumi}{(\alph{enumi})}
\renewcommand{\labelenumi}{\text{\theenumi}}

\usepackage{eurosym}

%SECTION
\titleformat{\section}[display]
{\huge\center\bfseries}
{}
{0pt}
{\Large}

\titleformat{\subsection}[display]
{\normalfont\raggedright\bfseries}
{}
{0pt}
{\Large}

\pagestyle{empty}


\begin{document}
\centering
{\Huge\bfseries Vorbereitung aufs Abi}\\
\vspace{0.5em}
{\LARGE\itshape Mathe-Crashkurs 2018}\\
\rule{0.7\textwidth}{0.5pt}
\vspace{1.5em}

\begin{justify}
  Den diesjährigen Mathe-Crashkurs am SGH bieten wir, die ehemaligen SGH-Schüler\\Simon König und Joshua Fabian (jetzt Informatik- und Physik-Studenten) an.\\
  Ihr erhaltet ein vollständiges Skript mit Erklärungen zu allen abiturrelevanten Themen und individulles Tutoring.\\
  Während des Kurses rechnen wir Aufgaben vor, erklären Themen, lösen Unklarheiten und helfen euch beim Lösen eigener Aufgaben.\\
  Die Teilnehmerzahl ist auf 20 begrenzt.
\end{justify}


\rule{0.5\textwidth}{0.5pt}
\vspace{1.5em}

\centering
\begin{tabular}{>{\bfseries\large}r p{0.6\textwidth}}
  Wann &
  In der zweiten Osterferienwoche, d.h.\newline
  \textbf{2.-6. April 2018 | 8:30 bis max. 18 Uhr}
  \\\\
  Was & \textbf{Komplettdurchlauf} des Stoffes von Klassen 11 und 12 anhand eines ausführlichen Skripts.
  \\\\

  Wo & Hier im Schickhardt, näheres nach Anmeldung.
  \\\\

  Preis & Der Kurs kostet 90\euro, hierbei sind Druckkosten für das 100-seitige Skript natürlich inbegriffen.
  \\\\
  Ablauf &
  \vtop{\vskip 0pt \vskip -\ht\strutbox
  \begin{tabular}{l p{0.35\textwidth}}
    8:30-ca. 12 Uhr:&Vorträge zur Wiederholung anhand von Beispielaufgaben\\
    bis ca. 13 Uhr:&Mittagspause\\
    nach 13 Uhr:&Aufgaben rechnen
  \end{tabular}\vskip -\dp\strutbox }
\end{tabular}
\\
\vspace{1.5em}
Für weitere Infos, E-Mail an: crashkurs-sgh@gmx.de.

\vspace{1em}
\rule{0.5\textwidth}{0.5pt}
\vspace{1.5em}

\vfill
\textbf{\large
Anmeldung unter:\\
\vspace{0.5em}
\Large crashkurs-sgh@gmx.de}

\vfill

\end{document}
