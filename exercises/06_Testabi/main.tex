\documentclass[a4paper, oneside]{article}
\usepackage[utf8]{inputenc}
\usepackage[ngerman]{babel}
\usepackage[top=2.5cm, bottom=3cm, outer=2.5cm, inner=2.5cm, heightrounded]{geometry}
\usepackage{graphicx}
\usepackage{morefloats}
\usepackage{wrapfig}
\usepackage{hyperref}
\usepackage{cite}
\usepackage{siunitx}
\usepackage[default]{sourcesanspro}
\usepackage[T1]{fontenc}
\usepackage{url}
\usepackage{marginnote}
\usepackage[font=footnotesize]{caption}
\usepackage{color}
\usepackage{xcolor}
\usepackage{multicol}
\usepackage[fleqn]{mathtools}
\usepackage{amssymb}
\usepackage{wrapfig}
\usepackage[noindentafter]{titlesec}
\usepackage{fancyhdr}
\usepackage{lastpage}
\usepackage{comment}

%% LÖSUNGEN ANZEIGEN
\newif\ifshow
%\showtrue
\showfalse

%%%SECTIONING
\renewcommand*{\marginfont}{\noindent\rule{0pt}{0.7\baselineskip}\footnotesize}

\newcommand{\aufgabe}[1]{\subsection{#1}}
\newcommand{\loesung}[1]{\subsubsection{#1}}

\renewcommand{\theenumi}{\alph{enumi})}
\renewcommand{\labelenumi}{\text{\theenumi}}

\newcounter{aufgabe}
%\newenvironment{lsg}{\loesung}{}
\ifshow
  \newenvironment{lsg}{\loesung}{}
\else
  \excludecomment{lsg}
\fi

\newenvironment{inhalt}
  {\paragraph{Inhalt des Übungsblatts:}\itemize\let\origitem\item}
  {\enditemize\vspace{2em}}

\newcommand{\R}{\ensuremath\mathbb{R}}
\newcommand{\N}{\ensuremath\mathbb{N}}
\newcommand{\Z}{\ensuremath\mathbb{Z}}
\newcommand{\LM}{\ensuremath\mathbb{L}}
\newcommand{\intd}{\ensuremath\mathrm{d}}
\newcommand{\e}{\ensuremath\mathrm{e}}
\renewcommand{\d}{\,\mathrm{d}}
\newcommand{\stf}[1]{\ensuremath \left[ #1 \right]}

\newcommand{\cas}{\hfill (CAS)}

\everymath{\displaystyle}

%Malpunkte
\mathcode`\*="8000
{\catcode`\*\active\gdef*{\cdot}}

%SECTION
\titleformat{\section}
{\clearpage\setcounter{aufgabe}{0}\vspace{1em}\Large\raggedright\bfseries}
{}
{0pt}
{}

\titleformat{\subsection}[runin]
{\stepcounter{aufgabe}\vspace{1em}\normalfont\raggedright\bfseries}
{A\theaufgabe: }
{0pt}
{\ }

\titleformat{\subsubsection}[runin]
{\normalfont\raggedright\bfseries}
{Lösung \theaufgabe: }
{0pt}
{\ }


%FANCYHDR
\pagestyle{fancy}
\lhead{\small Simon König\\ Joshua Fabian}
\rhead{\small Mathecrashkurs 2018}
\cfoot{Seite \thepage\thinspace von\thinspace\pageref{LastPage}}
\lfoot{}
\renewcommand{\headrulewidth}{0.5pt}
\renewcommand{\footrulewidth}{0pt}

\title{Mathe-Crashkurs 2018 - Übungsblatt}
\date{\today}
\author{Simon König, Joshua Fabian}


\chead{\Large Übungsblatt 6 - Probeklausur}
\begin{document}

\section{Pflichtteil}
\aufgabe{Analysis} Bilden Sie die erste Ableitung der Funktion $f$ mit $f(x) = (2x^2+5)e^{-2x}$

\aufgabe{Analysis} Berechnen Sie das Integral\\
$\int\limits_0^1(2x-1)^4 \intd x$

\aufgabe{Analysis}Lösen Sie die Gleichung $\e^x-\frac{3}{\e^x}+2=0$\\

\aufgabe{}
Graphanalyse
\loesung{}

\aufgabe{Analytische Geometrie}
Gegeben sei die Gerade $g: \vec x=\left(\begin{array}{c} 3 \\ -3 \\ 1 \end{array}\right) + t\cdot \left( \begin{array}{c} 1 \\ 2 \\ 0 \end{array} \right)$ und die Ebene $E = 2x_1-x_2+2x_3=2$.
\begin{enumerate}
  \item Zeigen Sie, dass $E$ und $g$ parallel sind, und berechnen Sie den Abstand von $g$ und $e$.
  \item Die Ebene $F$ ist orthogonal zu $E$ und enthält die Gerade $g$. Bestimmen Sie eine Gleichung der Schnittgeraden von $E$ und $F$.
\end{enumerate}

\aufgabe{Analytische Geometrie}
Von einem Senkrechten Kegel kennt man die Koordinaten der Spitze S, die Koordinaten eines Punktes P des Grundkreises sowie eine Koordinatengleichung der Ebene E, in der der Grundkreis liegt. Beschreiben Sie ein Verfahren, um den Mittelpunkt M und den Radius r des Grundkreises zu bestimmen.

\aufgabe{Stochastik}

\loesung{}


\section{Wahlteil Analysis}

\aufgabe{}

\aufgabe{}

\aufgabe{}

\section{Wahlteil Geometrie}

\aufgabe{}

\aufgabe{}

\aufgabe{}

\section{Wahlteil Stochastik}

\aufgabe{}

\aufgabe{}
\aufgabe{}
\aufgabe{}

\end{document}
